\documentclass[twosided]{memoir}
\usepackage{mybook}
\usepackage{bigfoot}
\begin{document}
\chapter{Conformal Mapping}
\section{The Logarithmic Function}
Consider the map defined by the function $w=\log z$. Let
\begin{equation}
	z=r\pbr*{\cos \phi +i\sin \phi }.
\end{equation}

Then
\begin{equation}\label{3.1.2}
\log z=\log r+\phi i.
\end{equation}
The function is infinitely multiple-valued. Select that branch for which
\begin{equation}\label{3.1.3}
-\pi <\phi <\pi.
\end{equation}
Thus the $z$-plane is cut along the negative axis of reals, and we are considering the single-valued function defined in this region, $S$, by equation \ref{3.1.2} and the inequalities \ref{3.1.3}. Let
\[
w=u+vi
.\] Then
\begin{equation}\label{3.1.4}
u=\log r, \quad v=\phi.
\end{equation}
The positive axis of reals in the $z$-plane, $\phi =0$, goes over into the entire axis of reals in the $w$-plane:
\begin{equation}\label{3.1.5}
-\infty<u<\infty, \quad v=0.
\end{equation}
An arbitrary ray,
\begin{equation}\label{3.1.6}
\phi =\alpha ,\quad -\pi <\alpha <\pi
\end{equation}
goes over into a parallel to the $u$-axis,
\begin{equation}\label{3.1.7}
v=\alpha .
\end{equation}
On the other hand, the unit circle, $r=1$, goes over into a segment of the axis of pure imaginaries:
\begin{equation}\label{3.1.8}
u=0, \quad -\pi <v<\pi.
\end{equation}
And an arbitrary circle,
\begin{equation}\label{3.1.9}
r=\rho,
\end{equation}
goes over into an equal line-segment, displaced horizontally:
\begin{equation}\label{3.1.10}
u=\log \rho ,\quad -\pi <v<\pi.
\end{equation}
It appears, then, that the region $S$ of the $z$-plane is carried over into a region $T$ of the $w$-plane, consisting of a strip bounded by the parallels,
\begin{equation}\label{3.1.11}
v=\pi ,\quad v=-\pi. 
\end{equation}
Moreover, the map is \emph{conformal}. We can bring this fact out suggestively by drawing a suitable network of lines in the two regions. Let the strip be divided into a large number of congruent strips, let us say 12. Then draw corresponding line segments equally spaced, beginning with \ref{3.1.8} and choosing $\rho $ in \ref{3.1.10} so that the distance between two successive line segments will be the same as the breadth of a strip. Thus for the first line segment to the right,
\[
\log \rho_1=\frac{2\pi }{12} =0.5236
\] and hence
\[
\rho =1.6881
.\] This means that the circle 
\[
\abs*{z} =1.6881
\] goes over into the line segment:
\[
u=0.5236, \quad -\pi <v<\pi 
.\] And similarly for the other line segments, for which
\[
\log \rho_n=0.5236n, \quad \rho _n=e^{0.5236n}, \quad n=0, \pm 1, \dots 
.\] Thus while the $u_n's$ form an \emph{arithmetic} series, the $\rho _n$'s form a \emph{geometric} series. The interior of the unit circle $\abs*{z} =1$ corresponds to the part of the strip $T$ to the left of the axis of pure imaginaries, and the images of the little squares in $T$ are the curvlinear quadrilaterals indicated in the figure. The exterior of the unit circle goes over into the part of the strip to the right.
\section*{Exercises}
\problem Into what figure is the quadrant of the unit circle which lies in the first quadrant carried by the above map?
\problem What region is the circular ring bounded by the circles $\abs*{z} =1$ and $\abs*{z=2} $ and cut open along the negative axis of reals, carried into?

\problem A rectangle in the $w$-plane is bounded by the lines
\[
u=-0.32,\quad u=1.8, \quad v=-0.25,\quad v=0
.\] Draw accurately the image in the $z$-plane.

\problem Plot accurately the point of the $w $-plane into which the point $z=-3-2i$ is transformed.
\problem Plot accurately the point of the $z$-plane into which the point $w=-0.5371-0.6873i$ goes.

\problem A plane area is bounded by two concentric circles of radii $2$ in. and $3$ in., and by two radii which make an angle of $45^\circ$ with each other. Show that it can be mapped conformally on a rectangle, and determine the ratio of the sides.

\section{The Function $w=z^\alpha $}

Consider the map defined by the function
\begin{equation}\label{3.2.1}
w=z^\alpha,
\end{equation}
where $\alpha $ is a positive real number. Let
\[
	z=r\pbr*{\cos \phi +i\sin \phi } ,\quad w=R\pbr*{\cos \psi +i\sin \psi } 
.\] Then
\[
	R\pbr*{\cos \psi +i\sin \psi } =r^{\alpha }\pbr*{\cos \alpha \psi +i\sin \alpha \psi } 
.\] Hence
\begin{equation}\label{3.2.2}
R=r^\alpha ,\quad \psi =\alpha \psi +2k\pi.
\end{equation}
Thus a circle about the origin, $z=0$, goes over into a circle about the origin, $w=0$.

Consider a sector of a circle:
\begin{equation}\label{3.2.3}
0\le r\le r_1,\quad 0\le \phi \le \phi_1.
\end{equation}
Let $\psi=\alpha \phi $, and assume that $\alpha \phi_1=\pi $, $1<\alpha $. Let $r_1=1$; then $R_1=1$. Thus a sector of the unit circle in the $z$-plane, whose angle is $\phi _1=\pi /\alpha $, is opened out like a fan on a semicircle. And yet, not wholly like a fan, for the points of the $z$-figure are drawn in toward the centre. If, for example, $\alpha =2$, the points on the circle $r=\frac{1}{2}$ go over into points on the circle $R=\frac{1}{4}$.

\section*{Exercises}
\problem Study in detail the case $\alpha =3$. Taking $5$ cm as the unit, draw accurately the two figures, dividing each of the angles into six equal parts. Recalling the corresponding figure in \S 1, choose as radii in the one figure:
\[
\rho _n=e^{-0.5236n}, \quad n=0,1,2,3,4
.\] 
\problem Examine the case $0<\alpha <1$. In particular, let $\alpha =\frac{1}{3}$:
\[
w=z^{\frac{1}{3}}
.\] How is this map related to the former map? Generalize.
\problem Study the case: $1<\alpha $,
\[
\alpha \phi _1=\pi ,\quad r_1=\infty
.\] Show that points inside the unit circle $\abs*{z} =1$ are drawn in nearer the origin, but points outside this circle are carried further away. Thus there is a stretching away from the unit circle, along the rays emanating from the origin.

\section{The Function $w=\sin ^{-1}z$}
The function 
\begin{equation}\label{3.3.1}
w=\sin ^{-1}z
\end{equation}
is defined by the equation:
\begin{equation}\label{3.3.2}
z=\sin w.
\end{equation}
Let
\[
w=u+vi,\quad z=x+yi
.\] Now,
\[
	\sin (u+vi)=\sin u\cos vi+\cos u\sin vi
.\] Recalling the formulas \ref{cos,sin} of Chapter 1, we have
\begin{align*}
	\cos vi&=\frac{e^{-v}+e^v}{2} =\ch v\\
	\sin vi&=\frac{e^{-v}-e^v}{2i} =i\sh v
\end{align*}
Hence \ref{3.3.2} becomes:
\begin{equation}\label{3.3.3}
x+yi=\sin u\ch v+i\cos u\sh v,
\end{equation}
and so
\begin{equation}\label{3.3.4}
x=\sin u\ch v,\quad y=\cos u \sh v.
\end{equation}
From these last equations we infer that
\begin{equation}\label{3.3.5}
\frac{x^2}{\ch ^2v}+\frac{y^2}{\sh^2 v}=1,\quad \frac{x^2}{\sin ^2u}-\frac{y^2}{\cos ^2u}=1.
\end{equation}
Thus it appears that the straight lines $v=\text{const} $ go over into ellipses with their foci in the points $x=\pm 1,\, y=0$; and the straight lines $u=\text{const} $ go over into hyperbolas with the same foci.

In particular, consider the strip in the $w$-plane bounded by the lines $u=\pm \pi /2$, with $v\ge 0$. A horizontal line segment
\[
-\frac{\pi}{2}\le u\le \frac{\pi}{2},\quad v=v_1>0,
\]goes over into the semi-ellipse
\[
\frac{x^2}{\ch^2v_1}+\frac{y^2}{\sh^2v_1}=1,
\] which lies in the upper half-plane:
\[
x=\ch v_1\sin u,\quad y=\sh v_1\cos u
.\] And similarly, a ray
\[
u=u_1, \quad -\frac{\pi}{2}<u_1<\frac{\pi}{2}; \quad v>0
,\] goes over into a half-branch of the hyperbola
\[
\frac{x^2}{\sin ^2u_1}-\frac{y^2}{\cos ^2u_1}=1
\]
\[
x=\sin u_1\ch v,\quad y=\cos u_1\sh v
.\] It is now easy to complete the map of the entire strip bounded by the lines $u=\pm \frac{\pi}{2}$. Reflect the map just constructed in each of the axes of reals. Thus the entire strip exclusive of the boundary is mapped on the entire $z$-plane exclusive of hte axis of reals to the right of $x=1$ and to the left of $x=-1$.

\section*{Exercises}
\problem Study in the same manner the map defined by the function
\[
w=\cos ^{-1}z
.\] 

\section{The Function $w=1 /z$}





\end{document}
