\chapter*{Preface}

The object of these Lectures is to give the student who has completed a course in Advanced Calculus, an introduction to the most important methods and results of Modern Analysis. The Cauchy theory, Weierstrass's analytic methods and results, Riemann's geometric and physical approach---this is the material out of which a systematic and harmonious treatment of this great branch of mathematics has grown. It was the Author's privilege to lecture on this subject for many years at Harvard University and, together with his colleagues there, to strive to make the treatment effective for the student of this First Course in Higher Analysis. 

I have chosen these words advisedly, for my colleague from after, reading them, says:---``But the theory of functions of real variables is the more elementary subject, and is needed for any adequate appreciation of the theory in the complex domain.''. True; a part of that theory is needed, and is provided for in these Lectures. But it must be borne in mind that the student is only just emerging from the Calculus, and it is reasonable to give him first what he can most readily receive. The theory of functions of real variables, if carried beyond its rudiments, soon loses contact, for the beginner, with the broader fields of analysis, geometry, and physics. It is these contacts, this broader knowledge, with which the beginner should become familiar before he specializes too closely in that great field, while for the student of Physics such specialization does not, at least at the present stage, come into consideration. And so I have restricted myself to those concepts and methods of that theory which are actually used in the present subject. The basal definitions and theorems are given in detail in the text; but a small amount of supplementary study in real analysis is suggested by specific references to the Author's \emph{Real Variables}. (\emph{Functions of Real Variables}, The University Press, The National University of Peking, 1936; referred to in the following pages as \emph{Real Variables}.)

Sufficient bibliographical references are included, to provide the student with the requisite historical background, and frequent references to the Author's \emph{Funktionentheorie} enable the reader to pursue a subject further. (\emph{Lehrbuch der Funktionentheorie}, vol. I, 5. ed. 1928, Theubner, Leipzig; referred to in the following pages as \emph{Funktionentheorie} I.)

After the rudiments of the subject have been treated, in Chapters I---VII, there is a wide choice of the closing topic. It might well be an application to the elliptic functions---and indeed this is the choice which the Author made in the \emph{Funktionentheorie}. Or, again, the linear total differential equations of the second order, with special reference to those which occur in Mathematical Physics, would be a highly appropriate subject. A large part of the treatment of differential equations in the \emph{Real Variables}, Chap. XII, can be carried over at once into the complex domain, and this, too, is a useful exercise for the student. Broader than any of these, however, is the Theory of the Potential Function, for in the hands of Riemann it yielded the fundamental theorem in conformal mapping, it opened up a new field in Algebraic Geometry, and it led to the automorphic functions. Furthermore, a study of the rudiments of the Logarithmic Potential forms an excellent introduction to the study of the Newtonian Potential Function.

To my colleague in Mathematics, Professor Kiang Tsai-Han, and to Dean Van Tsee-Chong, for their help in making the publication of these Lectures possible, I wish to express my hearty thanks. The services of my efficient Assistant, Mr. Sun Shu-Peng, in helping to prepare the manuscript for press and to carry about the typographical corrections, have been of the greatest value to me; I fell deep gratitude to him and the genuine interest he has uniformly shown in all these important details. Finally, my warm appreciation of the cooperation of the University Press in all that goes into the making of this book. 

\begin{flushright}
The National University of Peking\\January 1936
\end{flushright}
